\documentclass[journal, monochrome]{IEEEtran}
\usepackage[utf8]{inputenc}
\usepackage[monochrome]{color}
\usepackage{graphicx}
\usepackage{amsmath}
\usepackage{amssymb}
\usepackage{fancyvrb}
\usepackage{dsfont}
\usepackage{subfigure}
\usepackage{natbib}

\ifCLASSINFOpdf
\else
\fi
% graphicx was written by David Carlisle and Sebastian Rahtz. It is
% required if you want graphics, photos, etc. graphicx.sty is already
% installed on most LaTeX systems. The latest version and documentation can
% be obtained at: 
% http://www.ctan.org/tex-archive/macros/latex/required/graphics/
% Another good source of documentation is "Using Imported Graphics in
% LaTeX2e" by Keith Reckdahl which can be found as epslatex.ps or
% epslatex.pdf at: http://www.ctan.org/tex-archive/info/
%
% latex, and pdflatex in dvi mode, support graphics in encapsulated
% postscript (.eps) format. pdflatex in pdf mode supports graphics
% in .pdf, .jpeg, .png and .mps (metapost) formats. Users should ensure
% that all non-photo figures use a vector format (.eps, .pdf, .mps) and
% not a bitmapped formats (.jpeg, .png). IEEE frowns on bitmapped formats
% which can result in "jaggedy"/blurry rendering of lines and letters as
% well as large increases in file sizes.
%
% You can find documentation about the pdfTeX application at:
% http://www.tug.org/applications/pdftex





% *** MATH PACKAGES ***
%
%\usepackage[cmex10]{amsmath}
% A popular package from the American Mathematical Society that provides
% many useful and powerful commands for dealing with mathematics. If using
% it, be sure to load this package with the cmex10 option to ensure that
% only type 1 fonts will utilized at all point sizes. Without this option,
% it is possible that some math symbols, particularly those within
% footnotes, will be rendered in bitmap form which will result in a
% document that can not be IEEE Xplore compliant!
%
% Also, note that the amsmath package sets \interdisplaylinepenalty to 10000
% thus preventing page breaks from occurring within multiline equations. Use:
%\interdisplaylinepenalty=2500
% after loading amsmath to restore such page breaks as IEEEtran.cls normally
% does. amsmath.sty is already installed on most LaTeX systems. The latest
% version and documentation can be obtained at:
% http://www.ctan.org/tex-archive/macros/latex/required/amslatex/math/





% *** SPECIALIZED LIST PACKAGES ***
%
%\usepackage{algorithmic}
% algorithmic.sty was written by Peter Williams and Rogerio Brito.
% This package provides an algorithmic environment fo describing algorithms.
% You can use the algorithmic environment in-text or within a figure
% environment to provide for a floating algorithm. Do NOT use the algorithm
% floating environment provided by algorithm.sty (by the same authors) or
% algorithm2e.sty (by Christophe Fiorio) as IEEE does not use dedicated
% algorithm float types and packages that provide these will not provide
% correct IEEE style captions. The latest version and documentation of
% algorithmic.sty can be obtained at:
% http://www.ctan.org/tex-archive/macros/latex/contrib/algorithms/
% There is also a support site at:
% http://algorithms.berlios.de/index.html
% Also of interest may be the (relatively newer and more customizable)
% algorithmicx.sty package by Szasz Janos:
% http://www.ctan.org/tex-archive/macros/latex/contrib/algorithmicx/




% *** ALIGNMENT PACKAGES ***
%
%\usepackage{array}
% Frank Mittelbach's and David Carlisle's array.sty patches and improves
% the standard LaTeX2e array and tabular environments to provide better
% appearance and additional user controls. As the default LaTeX2e table
% generation code is lacking to the point of almost being broken with
% respect to the quality of the end results, all users are strongly
% advised to use an enhanced (at the very least that provided by array.sty)
% set of table tools. array.sty is already installed on most systems. The
% latest version and documentation can be obtained at:
% http://www.ctan.org/tex-archive/macros/latex/required/tools/


%\usepackage{mdwmath}
%\usepackage{mdwtab}
% Also highly recommended is Mark Wooding's extremely powerful MDW tools,
% especially mdwmath.sty and mdwtab.sty which are used to format equations
% and tables, respectively. The MDWtools set is already installed on most
% LaTeX systems. The lastest version and documentation is available at:
% http://www.ctan.org/tex-archive/macros/latex/contrib/mdwtools/


% IEEEtran contains the IEEEeqnarray family of commands that can be used to
% generate multiline equations as well as matrices, tables, etc., of high
% quality.


%\usepackage{eqparbox}
% Also of notable interest is Scott Pakin's eqparbox package for creating
% (automatically sized) equal width boxes - aka "natural width parboxes".
% Available at:
% http://www.ctan.org/tex-archive/macros/latex/contrib/eqparbox/





% *** SUBFIGURE PACKAGES ***
%\usepackage[tight,footnotesize]{subfigure}
% subfigure.sty was written by Steven Douglas Cochran. This package makes it
% easy to put subfigures in your figures. e.g., "Figure 1a and 1b". For IEEE
% work, it is a good idea to load it with the tight package option to reduce
% the amount of white space around the subfigures. subfigure.sty is already
% installed on most LaTeX systems. The latest version and documentation can
% be obtained at:
% http://www.ctan.org/tex-archive/obsolete/macros/latex/contrib/subfigure/
% subfigure.sty has been superceeded by subfig.sty.



%\usepackage[caption=false]{caption}
%\usepackage[font=footnotesize]{subfig}
% subfig.sty, also written by Steven Douglas Cochran, is the modern
% replacement for subfigure.sty. However, subfig.sty requires and
% automatically loads Axel Sommerfeldt's caption.sty which will override
% IEEEtran.cls handling of captions and this will result in nonIEEE style
% figure/table captions. To prevent this problem, be sure and preload
% caption.sty with its "caption=false" package option. This is will preserve
% IEEEtran.cls handing of captions. Version 1.3 (2005/06/28) and later 
% (recommended due to many improvements over 1.2) of subfig.sty supports
% the caption=false option directly:
%\usepackage[caption=false,font=footnotesize]{subfig}
%
% The latest version and documentation can be obtained at:
% http://www.ctan.org/tex-archive/macros/latex/contrib/subfig/
% The latest version and documentation of caption.sty can be obtained at:
% http://www.ctan.org/tex-archive/macros/latex/contrib/caption/




% *** FLOAT PACKAGES ***
%
%\usepackage{fixltx2e}
% fixltx2e, the successor to the earlier fix2col.sty, was written by
% Frank Mittelbach and David Carlisle. This package corrects a few problems
% in the LaTeX2e kernel, the most notable of which is that in current
% LaTeX2e releases, the ordering of single and double column floats is not
% guaranteed to be preserved. Thus, an unpatched LaTeX2e can allow a
% single column figure to be placed prior to an earlier double column
% figure. The latest version and documentation can be found at:
% http://www.ctan.org/tex-archive/macros/latex/base/



%\usepackage{stfloats}
% stfloats.sty was written by Sigitas Tolusis. This package gives LaTeX2e
% the ability to do double column floats at the bottom of the page as well
% as the top. (e.g., "\begin{figure*}[!b]" is not normally possible in
% LaTeX2e). It also provides a command:
%\fnbelowfloat
% to enable the placement of footnotes below bottom floats (the standard
% LaTeX2e kernel puts them above bottom floats). This is an invasive package
% which rewrites many portions of the LaTeX2e float routines. It may not work
% with other packages that modify the LaTeX2e float routines. The latest
% version and documentation can be obtained at:
% http://www.ctan.org/tex-archive/macros/latex/contrib/sttools/
% Documentation is contained in the stfloats.sty comments as well as in the
% presfull.pdf file. Do not use the stfloats baselinefloat ability as IEEE
% does not allow \baselineskip to stretch. Authors submitting work to the
% IEEE should note that IEEE rarely uses double column equations and
% that authors should try to avoid such use. Do not be tempted to use the
% cuted.sty or midfloat.sty packages (also by Sigitas Tolusis) as IEEE does
% not format its papers in such ways.


%\ifCLASSOPTIONcaptionsoff
%  \usepackage[nomarkers]{endfloat}
% \let\MYoriglatexcaption\caption
% \renewcommand{\caption}[2][\relax]{\MYoriglatexcaption[#2]{#2}}
%\fi
% endfloat.sty was written by James Darrell McCauley and Jeff Goldberg.
% This package may be useful when used in conjunction with IEEEtran.cls'
% captionsoff option. Some IEEE journals/societies require that submissions
% have lists of figures/tables at the end of the paper and that
% figures/tables without any captions are placed on a page by themselves at
% the end of the document. If needed, the draftcls IEEEtran class option or
% \CLASSINPUTbaselinestretch interface can be used to increase the line
% spacing as well. Be sure and use the nomarkers option of endfloat to
% prevent endfloat from "marking" where the figures would have been placed
% in the text. The two hack lines of code above are a slight modification of
% that suggested by in the endfloat docs (section 8.3.1) to ensure that
% the full captions always appear in the list of figures/tables - even if
% the user used the short optional argument of \caption[]{}.
% IEEE papers do not typically make use of \caption[]'s optional argument,
% so this should not be an issue. A similar trick can be used to disable
% captions of packages such as subfig.sty that lack options to turn off
% the subcaptions:
% For subfig.sty:
% \let\MYorigsubfloat\subfloat
% \renewcommand{\subfloat}[2][\relax]{\MYorigsubfloat[]{#2}}
% For subfigure.sty:
% \let\MYorigsubfigure\subfigure
% \renewcommand{\subfigure}[2][\relax]{\MYorigsubfigure[]{#2}}
% However, the above trick will not work if both optional arguments of
% the \subfloat/subfig command are used. Furthermore, there needs to be a
% description of each subfigure *somewhere* and endfloat does not add
% subfigure captions to its list of figures. Thus, the best approach is to
% avoid the use of subfigure captions (many IEEE journals avoid them anyway)
% and instead reference/explain all the subfigures within the main caption.
% The latest version of endfloat.sty and its documentation can obtained at:
% http://www.ctan.org/tex-archive/macros/latex/contrib/endfloat/
%
% The IEEEtran \ifCLASSOPTIONcaptionsoff conditional can also be used
% later in the document, say, to conditionally put the References on a 
% page by themselves.





% *** PDF, URL AND HYPERLINK PACKAGES ***
%
%\usepackage{url}
% url.sty was written by Donald Arseneau. It provides better support for
% handling and breaking URLs. url.sty is already installed on most LaTeX
% systems. The latest version can be obtained at:
% http://www.ctan.org/tex-archive/macros/latex/contrib/misc/
% Read the url.sty source comments for usage information. Basically,
% \url{my_url_here}.





% *** Do not adjust lengths that control margins, column widths, etc. ***
% *** Do not use packages that alter fonts (such as pslatex).         ***
% There should be no need to do such things with IEEEtran.cls V1.6 and later.
% (Unless specifically asked to do so by the journal or conference you plan
% to submit to, of course. )


% correct bad hyphenation here
\hyphenation{op-tical net-works semi-conduc-tor}


\begin{document}
%
% paper title
% can use linebreaks \\ within to get better formatting as desired
\title{Análisis de la velocidad del viento en Irlanda}
%
%
% author names and IEEE memberships
% note positions of commas and nonbreaking spaces ( ~ ) LaTeX will not break
% a structure at a ~ so this keeps an author's name from being broken across
% two lines.
% use \thanks{} to gain access to the first footnote area
% a separate \thanks must be used for each paragraph as LaTeX2e's \thanks
% was not built to handle multiple paragraphs
%

\author{\textbf{Autores:} Alejandro Magnorsky, Andrés Mata Suárez, Mariano Merchante \\[5px]
        Instituto Tecnológico de Buenos Aires}
        
% note the % following the last \IEEEmembership and also \thanks - 
% these prevent an unwanted space from occurring between the last author name
% and the end of the author line. i.e., if you had this:
% 
% \author{....lastname \thanks{...} \thanks{...} }
%                     ^------------^------------^----Do not want these spaces!
%
% a space would be appended to the last name and could cause every name on that
% line to be shifted left slightly. This is one of those "LaTeX things". For
% instance, "\textbf{A} \textbf{B}" will typeset as "A B" not "AB". To get
% "AB" then you have to do: "\textbf{A}\textbf{B}"
% \thanks is no different in this regard, so shield the last } of each \thanks
% that ends a line with a % and do not let a space in before the next \thanks.
% Spaces after \IEEEmembership other than the last one are OK (and needed) as
% you are supposed to have spaces between the names. For what it is worth,
% this is a minor point as most people would not even notice if the said evil
% space somehow managed to creep in.



% The paper headers
%\markboth{Template para los informes de SS, 2011}
%{Shell \MakeLowercase{\textit{et al.}}: Bare Demo of IEEEtran.cls for Journals}
% The only time the second header will appear is for the odd numbered pages
% after the title page when using the twoside option.
% 
% *** Note that you probably will NOT want to include the author's ***
% *** name in the headers of peer review papers.                   ***
% You can use \ifCLASSOPTIONpeerreview for conditional compilation here if
% you desire.




% If you want to put a publisher's ID mark on the page you can do it like
% this:
%\IEEEpubid{0000--0000/00\$00.00~\copyright~2007 IEEE}
% Remember, if you use this you must call \IEEEpubidadjcol in the second
% column for its text to clear the IEEEpubid mark.



% use for special paper notices
%\IEEEspecialpapernotice{(Invited Paper)}












% make the title area
\maketitle

\renewcommand{\abstractname}{Resumen}
\renewcommand{\IEEEkeywordsname}{Palabras clave}
\renewcommand{\refname}{Referencias}
\renewcommand{\tablename}{Tabla}

\begin{abstract}
\boldmath
Un ecosistema acuático en el que existen poblaciones de \textit{Tiburones Pintarrojo} y \textit{Salmones de Mar} puede estudiarse para analizar la convivencia entre
presas y predadores. Existe una forma de representar dicho ecosistema, y es mediante un modelo basado
en ecuaciones diferenciales. El siguiente artículo está orientado a analizar
dicho modelo. En base a datos de observaciones empíricas se realiza una simulación numérica 
para obtener funciones que representen las poblaciones de presas y predadores. Luego
se realiza un análisis de cómo el período de dichas funciones varía en función de los parámetros constantes de las ecuaciones diferenciales.
Se obtienen las poblaciones de equilibrio del sistema. Además, a fin de simplificar
el modelo, se propone una linealización del mismo.
\end{abstract}
% IEEEtran.cls defaults to using nonbold math in the Abstract.
% This preserves the distinction between vectors and scalars. However,
% if the journal you are submitting to favors bold math in the abstract,
% then you can use LaTeX's standard command \boldmath at the very start
% of the abstract to achieve this. Many IEEE journals frown on math
% in the abstract anyway.

% Note that keywords are not normally used for peerreview papers.
\begin{IEEEkeywords}
Convivencia presa-predador; \textit{Lotka-Volterra}; ecosistema acuático.
\end{IEEEkeywords}






% For peer review papers, you can put extra information on the cover
% page as needed:
% \ifCLASSOPTIONpeerreview
% \begin{center} \bfseries EDICS Category: 3-BBND \end{center}
% \fi
%
% For peerreview papers, this IEEEtran command inserts a page break and
% creates the second title. It will be ignored for other modes.
\IEEEpeerreviewmaketitle
\vspace{0.5cm}

\section{Introducción}

La manera en que se relacionan las presas y los predadores en un ecosistema natural
es interesante y ocupa un papel preponderante en la teoría ecologista. Su
análisis puede resultar útil en ciertas aplicaciones, tales como la predicción
de la posible extinción de una especie, así como la procreación intencional de
ciertos predadores para evitar el crecimiento descontrolado de una plaga. Es interesante saber entonces, dado un ecosistema, bajo qué condiciones
se establece una convivencia entre presas y predadores, o si alguna de las
dos poblaciones termina desapareciendo.\\

Existen varios modelos para representar la dinámica de un sistema presa-predador.
En este artículo se adopta el modelo dinámico de \textit{Lotka-Volterra-Ancona}\footnote{Alfred James Lotka (1880-1949) fue un matemático estadounidense, especializado en estadística. 
Vito Volterra (1860-1940) fue un matemático y físico italiano. Umberto d’Ancona, cuñado de Vito Volterra, era un pescador italiano.},
que constituye un gran avance en el estudio de estas poblaciones ocurrido entre los años
1925 y 1928.\\

En la sección \ref{section:development}, se presentan las ecuaciones diferenciales
que constituyen el modelo mencionado en el párrafo anterior y se lo clasifica. Se realiza una simulación numérica para, en base al modelo y a observaciones empíricas, obtener curvas que 
representen a las poblaciones de presas y predadores. Se grafica la trayectoria del \textit{espacio de fases}. Se analiza el comportamiento de las curvas en base a los parámetros constantes 
de las ecuaciones diferenciales y se encuentran las poblaciones de equilibrio. Finalmente, se realiza una linealización del sistema.\\

La sección \ref{section:results} contiene los resultados obtenidos luego de realizar los distintos análisis de la sección \ref{section:development}. Además, se realizan observaciones sobre los mismos.\\

%-------------------------------------------------------------------------
\section{Desarrollo}
\label{section:development}

\par
Para cada una de las estaciones meteorológicas, se desea ajustar los datos a una función de la forma:

\begin{equation}
v(t) = A_{0} + A_{1}cos(2\pi f_{1}t) + B_{1}sen(2\pi f_{1}t)
\label{equation:model}
\end{equation}

donde $t$ es el tiempo en días, $v(t)$ es la velocidad del viento en el instante $t$ y $f_{1} = \frac{1}{365,25} dia^{-1}$. \\
$v(t)$ es lineal en función de $A_{0}$, $A_{1}$ y $B_{1}$ y se puede expresar como $v(t) = A_{0}f_{1}(t) + A_{1}f_{2}(t) + B_{1}f_{3}(t)$ donde $f_{1}(t) = 1$, $f_{2}(t) = cos(2\pi f_{1}t)$ y $f_{3}(t) = sen(2\pi f_{1}t)$. \\
Para resolver el problema de ajuste se utiliza el método de cuadrados mínimos. El objetivo entonces es encontrar $A_{0}$, $A_{1}$ y $B_{1}$ tal que 
$\displaystyle\sum_{\substack{k=1}}^{n} (v(t_{k})- v_{k})^{2} $ sea mínimo. \\
Si definimos la matriz A y los vectores $\vec{x}$ y $\vec{b}$ como:
\begin{equation}
A = \left(\begin{array}{ccc}
f_{1}(t_{1}) & f_{2}(t_{1}) & f_{3}(t_{1}) \\
f_{1}(t_{2}) & f_{2}(t_{2}) & f_{3}(t_{2}) \\
\vdots & \vdots & \vdots \\
f_{1}(t_{n}) & f_{2}(t_{n}) & f_{3}(t_{n}) \\
\end{array} \right) \qquad
\vec{x} = \left(\begin{array}{c}
A_{0} \\
A_{1} \\
B_{1} \\
\end{array} \right) \qquad
\vec{b} = \left(\begin{array}{c}
v_{1} \\
v_{2} \\
\vdots \\
v_{n} \\
\end{array} \right)
\end{equation}
entonces el objetivo puede expresarse como encontrar $\vec{x}$ tal que $||A\vec{x} - {b}||^{2}$ sea mínimo. \\
\begin{equation}
A = QR = (Q_{1} \, Q_{2}) \left( \begin{array}{c}
R_{1} \\
0 \\
\end{array} \right)
\end{equation}
donde A tiene dimensiones $n$x$3$, $Q_{1}$ de $n$x$3$, $Q_{2}$ de $n$x$(n-3)$ y $R_{1}$ de $3$x$3$.
La solución se obtiene de:
\begin{eqnarray}
A\vec{x} & = & \vec{b} \\
R_{1}\vec{x} & = & Q_{1}^{T}\vec{b} 
\end{eqnarray}


\vspace{0.5cm}
%-------------------------------------------------------------------------
\section{Resultados y conclusiones}
\label{section:results}

\subsection{Curvas obtenidas mediante simulaciones}

El valor de la constante $\mu$ hallado mediante la ecuación \ref{equation:lsSystemSolution} es $1.87 \text{ yr}^{-1}$, y el de $K$ es $8.22 \text{ yr}^{-1}$. Dado que el valor de $\mu$ es mayor que $1 \text{ yr}^{-1}$, 
en ausencia de presas la población de predadores desaparece en menos de un año.\\

Mediante la obtención de los residuos de la aproximación por cuadrados mínimos, se evalúa qué tan certero es el valor de $\mu$ hallado. La figura \ref{figure:residuals} muestra dichos residuos. 
Puede verse que para varios valores del tiempo, los residuos son cercanos a $0$. No obstante, particularmente para el valor $t = 8.8 \text{ yr}$, se tiene un residuo de $r = 2.68$. Dado que éste es muy distinto a los valores esperados, se evalúa si constituye un \textit{outlier} en la distribución de residuos encontrada. La media muestral es $m = 0$, y el desvío estándar es $\sigma = 0.67$. $r$ es mayor a $3 \sigma = 2.01$, por lo tanto es considerado un \textit{outlier}. Sin embargo, como es el único \textit{outlier} presente, puede considerarse que la aproximación es aceptable.\\



A partir del valor de $\mu$ hallado, y utilizando los valores conocidos de las demás constantes, se utiliza el método de \textit{Runge-Kutta} de cuarto orden \citep{mathews} para resolver numéricamente las ecuaciones 
\ref{equation:preys} y \ref{equation:predators}. El paso utilizado fue $h = 0.01 \text{ yr}$. La figura \ref{figure:simmulatedPopulations} muestra las poblaciones obtenidas de esta manera. Puede observarse que las curvas periódicas obtenidas 
son similares a la de la figura \ref{figure:populations}, correspondiente al conjunto de datos; no obstante, en las curvas simuladas, no se alcanzan los picos de las curvas reales.\\


Las poblaciones obtenidas mediante la simulación pueden utilizarse para construir el espacio de fases dado por la ecuación \ref{equation:phaseSpace}. La misma puede observarse en la figura \ref{figure:phaseSpace}. 
La curva obtenida es cerrada, hecho que indica que los períodos de las poblaciones $x$ e $y$ son iguales.\\

Es interesante tener la posibilidad de conocer cómo crecen y decrecen las poblaciones a medida que pasa el tiempo. En ausencia de las curvas de las figuras \ref{figure:populations} y 
\ref{figure:simmulatedPopulations}, esto es imposible debido a que la figura \ref{figure:phaseSpace} es una trayectoria en la que no está involucrada la variable tiempo. Sin embargo, 
puede calcularse el sentido de esta trayectoria y, así, tener información de la evolución dinámica de las poblaciones. Para determinarlo se obtiene, en base a las poblaciones simuladas, 
el valor de $\dot{y}$ en el par $(x, y)$ en el cual $x$ es máximo. Este par corresponde a $x = 154.13$ e $y = 76.10$, y en este punto se tiene $\dot{y} = 267.97 \text{ yr}^{-1}$. Dado que este valor es positivo,
la población de predadores crece, con lo que el sentido resulta ser antihorario.

\subsection{Períodos de las poblaciones}

Realizando simulaciones utilizando los parámetros dados y variando $\mu$, se obtiene que
dichos períodos decrecen hasta aproximadamente $\mu = 10 \text{ yr}^{-1}$, en donde alcanzan un mínimo luego del cual comienzan a crecer. En este mínimo, los períodos son de $2.37 \text{ yr}$. 
Similarmente, realizando simulaciones utilizando los parámetros dados y el $\mu$ estimado (utilizando la ecuación \ref{equation:lsSystemSolution}) y variando el valor del parámetro $\lambda$, se obtienen
los valores de los períodos en función de $\lambda$. Nuevamente, los períodos son decrecientes hasta un mínimo, luego del cual comienzan a crecer. El mínimo se produce para $\lambda = 7 \text{ yr}^{-1}$
aproximadamente, en donde los períodos son de $3.37 \text{ yr}$. En la figura \ref{figure:periods} se aprecia la forma en que varía el período para distintos valores de $\mu$, y para distintos valores de $\lambda$.


\subsection{Desfasaje en las curvas}

Dada la periodicidad de las curvas que representan a las poblaciones, se realizan simulaciones con el motivo de obtener una dependencia 
entre el desfasaje $\phi$ entre las poblaciones de presas y predadores, y el parámetro $a$. En otras palabras, se busca determinar $\phi$ tal que $\phi = \phi(a)$.\\

Para valores de $a$ variando desde $0.02 \text{ yr}^{-1}$ hasta $1 \text{ yr}^{-1}$ (se considera innecesario hacer la evaluación para valores más cercanos a $0$, debido a que eso implicaría la inexistencia de encuentros 
perjudiciales para las presas), se obtiene la gráfica de $\phi$ en función de $a$, detallada en la figura \ref{figure:phi}.
En base a ésta, estimando geométricamente el comportamiento de la gráfica, se propone que el desfasaje obedezca una ecuación de la forma:
\begin{equation}
\phi(a) = \frac{1}{A (a + d)^2 + B (a + d) + C}
\label{equation:fi_eq2}
\end{equation}
donde $A$, $B$, $C$ y $d$ son constantes reales.\\

Puede realizarse una deducción empírica de tales constantes con el objeto de aproximar mediante una ecuación, de la mejor manera posible, el gráfico expuesto en la figura \ref{figure:phi}. Haciendo esto, se 
tiene que para $A = 344.83 \text{ yr}$, $B = 27.59$, $C = 6.90 \text{ yr}^{-1}$ y $d = -0.35 \text{ yr}^{-1}$, el gráfico de $\phi(a)$ es el observado en la figura \ref{figure:phiSimApprox}.
Se decide, entonces, aproximar $\phi$ mediante la ecuación \ref{equation:fi_eq2} con estos valores de las constantes reales.



\subsection{Poblaciones de equilibrio}

El punto de equilibrio para el cual las poblaciones se mantienen constantes se obtiene reemplazando los datos $\lambda$, $\mu$, $a$ y $b$ por sus respectivos valores en la ecuación \ref{equation:eq1}, 
lo cual resulta en $(x_{eq}, y_{eq}) = (53.52, 75)$. \\

Habiendo calculado, de manera empírica, cuáles son las condiciones iniciales tales que ambas poblaciones perduren en el tiempo, se obtiene el gráfico de la figura \ref{figure:prob2b}. Del mismo puede deducirse que las poblaciones con gran cantidad de presas y pocos predadores, o con pocas presas y muchos predadores, tienen mayor probabilidad de sobrevivir que el caso en el que ambas poblaciones son grandes. Se confirma que, como lo indican las ecuaciones \ref{equation:preys} y \ref{equation:predators}, si hay muchas presas y muchos predadores, la velocidad de crecimiento de los predadores es muy alta y además las presas tienen una velocidad negativa de crecimiento considerable, lo que lleva a la extinción de las presas y por consiguiente a la extinción de los predadores.


\subsection{Linealización}

La ecuación \ref{equation:clearLinearization}, correspondiente a la linealización del sistema, puede reformularse de la siguiente manera:
\begin{equation}
\dot{x} \approx -\frac{a \mu}{b} \left( y - \frac{\lambda}{a}\right)
\label{equation:preysLinear2}
\end{equation}
\begin{equation}
\dot{y} \approx \frac{b \lambda}{a}\left(x - \frac{\mu}{b}\right)
\label{equation:predatorsLinear2}
\end{equation}

Es interesante notar que, a puntos cercanos a $X_{eq} = (\frac{\mu}{b}, \frac{\lambda}{a})$ la variación de la población de las presas es directamente proporcional a la diferencia de población de los predadores con respecto a la población de equilibrio, y viceversa. 

%-------------------------------------------------------------------------




% trigger a \newpage just before the given reference
% number - used to balance the columns on the last page
% adjust value as needed - may need to be readjusted if
% the document is modified later
%\IEEEtriggeratref{8}
% The "triggered" command can be changed if desired:
%\IEEEtriggercmd{\enlargethispage{-5in}}

% references section

% can use a bibliography generated by BibTeX as a .bbl file
% BibTeX documentation can be easily obtained at:
% http://www.ctan.org/tex-archive/biblio/bibtex/contrib/doc/
% The IEEEtran BibTeX style support page is at:
% http://www.michaelshell.org/tex/ieeetran/bibtex/
%\bibliographystyle{IEEEtran}
%\bibliographystyle{authordate}
% argument is your BibTeX string definitions and bibliography database(s)
%\bibliography{IEEEabrv,../bib/paper}
%
% <OR> manually copy in the resultant .bbl file
% set second argument of \begin to the number of references
% (used to reserve space for the reference number labels box)


\begin{thebibliography}{1}

\bibitem[Berryman(1992)]{berryman}
	Berryman, Alan A.
	``The origins and evolution of predator-prey history'',
	Ecological Society of America,
	1992

\vspace{0.2cm}

\bibitem[Mathews y Fink(1992)]{mathews}
	Mathews, John H.,
	Fink, Kurtis D.,
	``Numerical Methods Using MATLAB'',
	Prentice Hall,
	1999
	
\end{thebibliography}

\end{document}


